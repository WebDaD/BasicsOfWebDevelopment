\section{Das Internet}
Das Internet (wörtlich etwa „Zwischennetz“ oder „Verbundnetz“, von engl.: interconnected Networks: „untereinander verbundene Netzwerke“) ist ein weltweites Netzwerk bestehend aus vielen Rechnernetzwerken, durch das Daten ausgetauscht werden. Es ermöglicht die Nutzung von Internetdiensten wie E-Mail, Telnet, Usenet, Dateiübertragung, WWW und in letzter Zeit zunehmend auch Telefonie, Radio und Fernsehen. 
Im Prinzip kann dabei jeder Rechner weltweit mit jedem anderen Rechner verbunden werden. Der Datenaustausch zwischen den einzelnen Internet-Rechnern erfolgt über die technisch normierten Internetprotokolle. Die Technik des Internet wird durch die RFCs der IETF (Internet Engineering Task Force) beschrieben. 

Umgangssprachlich wird „Internet“ häufig synonym zum World Wide Web verwendet, da dieses einer der meistgenutzten Internetdienste ist, und im wesentlichen zum Wachstum und der Popularität des Mediums beigetragen hat. 
Im Gegensatz dazu sind andere Mediendienste, wie Telefonie, Fernsehen und Radio erst kürzlich über das Internet erreichbar und haben immer noch ihre eigenen Netzwerke.

Das Internet ging aus dem im Jahr 1969 entstandenen ARPANET hervor, einem Projekt der Advanced Research Project Agency (ARPA) des US-Verteidigungsministeriums. Es wurde zur Vernetzung von Universitäten und Forschungseinrichtungen benutzt. Ziel des Projekts war zunächst, die knappen Rechenkapazitäten sinnvoll zu nutzen, erst in den USA, später weltweit. 
Die anfängliche Verbreitung des Internets ist eng mit der Entwicklung des Betriebssystems Unix verbunden. Nachdem das Arpanet im Jahr 1982 TCP/IP adaptierte, begann sich auch der Name Internet durchzusetzen.

Nach einer weit verbreiteten Legende bestand das ursprüngliche Ziel des Projektes vor dem Hintergrund des Kalten Krieges in der Schaffung eines verteilten Kommunikationssystems, um im Falle eines Atomkrieges eine störungsfreie Kommunikation zu ermöglichen. In Wirklichkeit wurden vorwiegend zivile Projekte gefördert, auch wenn die ersten Knoten von der Advanced Research Projects Agency (ARPA) finanziert wurden.

Die wichtigste Applikation in den Anfängen war die E-Mail. Bereits im Jahr 1971 überstieg das Gesamtvolumen des E-Mail-Verkehrs das Datenvolumen, das über die anderen Protokolle des Arpanet, das Telnet und FTP abgewickelt wurde.
Rasanten Auftrieb erhielt das Internet seit dem Jahr 1993 durch das World Wide Web, kurz WWW, als der erste grafikfähige Webbrowser namens Mosaic veröffentlicht und zum kostenlosen Download angeboten wurde. 
Das WWW wurde im Jahr 1989 im CERN (bei Genf) von Tim Berners-Lee entwickelt. Schließlich konnten auch Laien auf das Netz zugreifen, was mit der wachsenden Zahl von Nutzern zu vielen kommerziellen Angeboten im Netz führte.
Der Webbrowser wird deswegen auch als die „Killerapplikation“ des Internet bezeichnet. Das Internet ist ein wesentlicher Katalysator der Digitalen Revolution.
Im Jahr 1990 beschloss die US-amerikanische National Science Foundation, das Internet für kommerzielle Zwecke zu nutzen, wodurch es über die Universitäten hinaus öffentlich zugänglich wurde.
Neue Techniken verändern das Internet und ziehen neue Benutzerkreise an: IP-Telefonie, Groupware wie Wikis, Blogs, Breitbandzugänge (zum Beispiel für Vlogs und Video-on-Demand), Peer-to-Peer-Vernetzung (vor allem für File Sharing) und Online-Spiele (z. B. Rollenspiele, Taktikshooter, …).

Das rasante Wachstum des Internets sowie Unzulänglichkeiten für immer anspruchsvollere Anwendungen bringen es jedoch möglicherweise in Zukunft an seine Grenzen, so dass inzwischen Forschungsinitiativen begonnen haben, das Internet der Zukunft zu entwickeln.

Das Internet besteht aus Netzwerken unterschiedlicher administrativer Verwaltung, welche zusammengeschaltet werden. Darunter sind hauptsächlich:

Providernetzwerke, an die die Rechner der Kunden eines Internetproviders angeschlossen sind,
Firmennetzwerke (Intranets), über welche die Computer einer Firma verbunden sind, sowie
Universitäts- und Forschungsnetzwerke.

Physikalisch besteht das Internet im Kernbereich (in den Backbone-Netzwerken) sowohl kontinental als auch interkontinental hauptsächlich aus Glasfaserkabeln, die durch Router zu einem Netz verbunden sind. 
Glasfaserkabel bieten eine enorme Übertragungskapazität und wurden vor einigen Jahren zahlreich sowohl als Land- als auch als Seekabel in Erwartung sehr großen Datenverkehr-Wachstums verlegt. Da sich die physikalisch mögliche Übertragungsrate pro Faserpaar mit fortschrittlicher Lichteinspeisetechnik (DWDM) aber immens vergrößerte, besitzt das Internet hier zur Zeit teilweise Überkapazitäten. Schätzungen zufolge wurden im Jahr 2005 nur etwa 3 % der zwischen europäischen oder US-amerikanischen Städten verlegten Glasfasern benutzt. Auch Satelliten und Richtfunkstrecken sind in die globale Internet-Struktur eingebunden, haben jedoch einen geringen Anteil.

Auf der sogenannten letzten Meile, also bei den Hausanschlüssen, werden die Daten oft auf Kupferleitungen von Telefon- oder Fernsehanschlüssen und vermehrt auch über Funk, mittels WLAN oder UMTS, übertragen. 
Glasfasern bis zum Haus (FTTH) sind in Deutschland noch nicht sehr weit verbreitet. Privatpersonen greifen auf das Internet entweder über einen Schmalbandanschluss, zum Beispiel per Modem oder ISDN (siehe auch Internet by Call), oder über einen Breitbandzugang, zum Beispiel mit DSL, Kabelmodem oder UMTS, eines Internetproviders zu. Firmen oder staatliche Einrichtungen sind häufig per Standleitung mit dem Internet verbunden, wobei Techniken wie Kanalbündelung, ATM, SDH oder - immer häufiger - Ethernet in allen Geschwindigkeitsvarianten zum Einsatz kommen.

In privaten Haushalten werden oft Computer zum Abrufen von Diensten ans Internet angeschlossen, die selbst wenige oder keine solche Dienste für andere Teilnehmer bereitstellen und welche nicht dauerhaft erreichbar sind. 
Solche Rechner werden als Client-Rechner bezeichnet. Server dagegen sind Rechner, welche in erster Linie Internetdienste bereitstellen. 


Sie stehen meistens in sogenannten Rechenzentren, sind dort schnell angebunden und die Räumlichkeiten sind gegen Strom- und Netzwerkausfall sowie Einbruch und Brand gesichert. Peer-to-Peer-Anwendungen versetzen auch obige Client-Rechner in die Lage zeitweilig selbst Dienste anzubieten, die sie bei anderen Rechnern dieses Verbunds abrufen und so wird hier die strenge Unterscheidung des Client-Server-Modells aufgelöst.

An Internet-Knoten werden viele verschiedene Backbone-Netzwerke über leistungsstarke Verbindungen und Geräte (Router und Switches) miteinander vernetzt. Am DE-CIX in Frankfurt am Main, dem größten deutschen Austauschpunkt dieser Art, sind es beispielsweise mehr als hundert Netzwerke. 
Eine solche Übergabe von Datenverkehr zwischen getrennten administrativen Bereichen, sogenannten autonomen Systemen, kann auch an jedem anderen Ort geschaltet werden, es ist nur oft wirtschaftlich sinnvoll Internet-Knoten zu erschließen. Da in der Regel ein Internetprovider nicht alle anderen Provider auf diese Art treffen kann, benötigt er selbst mindestens einen Provider, der für ihn den verbleibenden Datenverkehr gegen Bezahlung zustellt. 
Es gibt derzeit zehn sehr große, sogenannte Tier-1-Provider, die ihren gesamten Datenverkehr auf Gegenseitigkeit abwickeln können.

Da das Arpanet als dezentrales Netzwerk möglichst ausfallsicher sein sollte, wurde schon bei der Planung beachtet, dass es keinen Zentralrechner, keinen zentralen Internet-Knoten sowie keinen Ort geben sollte, an dem alle Verbindungen zusammenlaufen. Diese geplante Dezentralität wurde jedoch auf der administrativen Ebene des Internet nicht durchgängig eingehalten. 
Die Internet Corporation for Assigned Names and Numbers (ICANN), die hierarchisch höchste zuständige Organisation für die Vergabe von IP-Adressen, die Koordination des Domain Name Systems (DNS) und der dafür nötigen Root-Nameserver-Infrastruktur, sowie die Festlegung anderer Parameter der Internetprotokollfamilie, welche weltweite Eindeutigkeit verlangen, untersteht wenigstens indirekt dem Einfluss des US-Wirtschaftsministeriums. 
Um diesen Einfluss zumindest auf das DNS einzugrenzen, wurde das in erster Linie europäische Open Root Server Network aufgebaut, das jedoch mit dem Jahresende 2008 aus nachlassendem Interesse wieder abgeschaltet wurde. 

Die netzartige Struktur sowie die Heterogenität des Internets tragen zu einer hohen Ausfallsicherheit bei. Für die Kommunikation zwischen zwei Nutzern existieren meistens mehrere mögliche Wege und erst bei der tatsächlichen Datenübertragung wird entschieden, welcher benutzt wird. 
Dabei können zwei hintereinander versandte Datenpakete, beziehungsweise eine Anfrage und die Antwort, je nach Auslastung und Verfügbarkeit verschiedene Pfade durchlaufen. Deshalb hat der Ausfall einer physikalischen Verbindung im Kernbereich des Internets meistens keine schwerwiegenden Auswirkungen, sondern kann durch die Verwendung alternativer Kommunikationswege ausgeglichen werden. 

Im Bereich der Katastrophenforschung werden flächendeckende Missbräuche oder Ausfälle des Internets, sog. D-Gefahren, sehr ernst genommen. Ein Zusammenbruch des Internets oder einzelner Teile hätte weitreichende Folgen.